% ********** Chapter 4 **********
\chapter{Photos Service}
\label{sec:chapter4}

This service allows search and manage of photos that are hosted in \textbf{SAPO Photos}. 

To interact with the service are available a SOAP interface and a HTTP-JSON interface.

\section{HTTP}
\label{sec:chapter4:http}

This section describes the HTTP-JSON interface.

In all the samples of this chapter it is used \icode|ESBUsername| and \icode|ESBPassword| to do the authentication. But as mentioned in \autoref{sec:chapter3:http} alternatively you can provided \icode|ESBToken|. 

\subsection{ImageCreate}
\label{sec:chapter4:photos:http:imagecreate}

The operation \textbf{ImageCreate} is composed by two steps. First you have to send the meta data associated with the photo through a HTTP POST to \textit{https://services.sapo.pt/Photos/ImageCreate}. Second you have to do a HTTP POST to \textit{http://fotos.sapo.pt/uploadPost.html} with the photo file.

\subsubsection{POST the meta data of the photo}
\label{sec:chapter4:photos:http:imagecreate:metadata}

In \autoref{listpostmetadata} is presented a sample request of the POST of the meta data of the photo.

For the sake of simplicity there are only provided the title and the tags of the image. It's recommended that when you submit a new photo you provide at least this attributes.

The complete list of Image attributes can be found at \cite{photosimage}. But if you go forward a few pages and look carefully to \autoref{ImageCreateResponse} you can see the structure of the \icode|Image| object.

\begin{lstlisting}[label=listpostmetadata,caption=POST the meta data of the photo]
POST https://services.sapo.pt/Photos/ImageCreate?json=true&ESBUsername={username}&ESBPassword={password} HTTP/1.1
Content-Type: application/json
Authorization: ESB AccessKey={accessKey}
Content-Length: 48
Host: services.sapo.pt

{"image":{"title":"windows8","tags":"windows8"}}
\end{lstlisting} 

The service response will have a similar structure to the one in \autoref{ImageCreateResponse}. In the body of the response figures an object that has a ImageCreateResponse attribute. The value attribute is a ImageCreateResult object that as two attributes: image and result.

The image is a instance of \icode|Image| type, which has the meta data supplied and another server generated fields, like \icode|uid| that identifies the photo.

You can see if the request was accepted by the server looking to the \icode|ok| attribute of the \icode|result| object. The \icode|{photo_token}| attribute is the token that you will have to send along with the photo file. 

Note that \icode|{photo_token}| is place holder and \icode|{...}| is censuring sensitive data like usernames and photo ids.

\begin{lstlisting}[label=ImageCreateResponse,caption=Response of the HTTP POST with the meta data]
HTTP/1.1 200 OK
Content-Type: application/json; charset=utf-8
Date: Tue, 18 Sep 2012 10:14:54 GMT
Content-Length: 1014

{
   "ImageCreateResponse":{
      "ImageCreateResult":{
         "image":{
            "active":"true",
            "creationDate":"2012-09-18 11:00:43",
            "innapropriate":"false",
            "m18":"false",
            "password":null,
            "pending":"true",
            "rating":"0",
            "subtitle":null,
            "synopse":null,
            "tags":"sapo",
            "title":"sapo",
            "uid":"{...}",
            "url":"http://fotos.sapo.pt/{...}/fotos/?uid={...}",
            "user":{
               "avatar":"http://imgs.sapo.pt/sapofotos/{...}/imgs/avatar.jpg",
               "url":"http://fotos.sapo.pt/{...}/perfil",
               "username":"{...}"
            },
            "visualizations":"0"
         },
         "result":{
            "ok":"true"
         },
         "token":"{photo_token}"
      }
   }
}
\end{lstlisting}

\subsubsection{POST the photo file}
\label{sec:chapter4:photos:http:imagecreate:file}

The second and final step of the upload of the photo is a \icode|multipart| POST.

The structure of the request can be seen in \autoref{listpostphotobytes}.

Note that \icode|{photo_token}| and \icode|{photo_bytes}| are place holders. The first one is the token returned by the server in the \icode|ImageCreateResult| object. The other is the bytes of the photo file. 

\begin{lstlisting}[label=listpostphotobytes,caption=HTTP GET using \icode|ESBToken|]
POST http://fotos.sapo.pt/uploadPost.html HTTP/1.1
Content-Type: multipart/form-data; boundary="imgboundary"
Host: fotos.sapo.pt
Content-Length: 75195

--imgboundary
Content-Disposition: form-data; name="token"

{photo_token}
--imgboundary
Content-Type: image/png
Content-Disposition: form-data; name="photo"; filename="windows8.png"

{photo_bytes}
\end{lstlisting} 

If all goes as it should in the response you will get a response with \textbf{XML} content where you can find a \icode|Result| tag with "SUCCESS" in the content. Otherwise you will get a response with a smaller \textbf{XML} document, that has in the \icode|Result| tag the error code. You can see the complete error list at \cite{imagecreate}.

In \autoref{uploadPostResponseSuccess} you can see a sample response body. In this case the photo was successfully submitted. So in the \textbf{XML} document will be present all the \icode|views| generated. Each \icode|view| has a URI (\icode|url| tag) to the photo file. Along with the URI is also provided the with and the height of the photo.

\begin{lstlisting}[label=uploadPostResponseSuccess,caption=Photo file upload response body]
<?xml version="1.0"?>
<uploadPost>
 <Ok/>
 <Result>SUCCESS</Result>
 <views>
  <view>
   <size>large</size>
   <requestWidth>1600</requestWidth>
   <requestHeight>1200</requestHeight>
   <url>http://{...}.png</url>
  </view>
  (...)
 </views>
</uploadPost>
\end{lstlisting}

In case of error, for instance if you provide an invalid token the body of the response will look like the one in \autoref{listpostinvalidtoken}.

\textbf{Important:} Note that in case of error the \icode|UploadPost| tag begin with capital letter.

\begin{lstlisting}[label=listpostinvalidtoken,caption=Response body in case of invalid token]
<?xml version="1.0" encoding="utf-8" ?>
<UploadPost>
  <Result>INVALID_TOKEN</Result>
  <Error>Invalid token</Error>
</UploadPost> 
\end{lstlisting}

\subsection{ImageGetListBySearch}
\label{sec:chapter4:photos:http:imageGetListBySearch}

\begin{lstlisting}[label=imageGetListBySearch,caption=ImageGetListBySearch sample request]
GET http://services.sapo.pt/Photos/ImageGetListBySearch?string=windows8&page=1&datefrom=2012-09-10&dateto=2012-09-12&json=true&ESBUsername={username}&ESBPassword={pass} HTTP/1.1
Authorization: ESB AccessKey={accessKey}
Host: services.sapo.pt
\end{lstlisting}


\begin{lstlisting}[label=imageGetListBySearchResponse,caption=ImageGetListBySearch sample response body]
{
   "ImageGetListBySearchResponse":{
      "ImageGetListBySearchResult":{
         "images":{
            "image":[
               {
                  "active":"true",
                  "creationDate":"2011-09-14 22:56:00",
                  "innapropriate":"false",
                  "m18":"false",
                  "password":null,
                  "pending":"false",
                  "subtitle":null,
                  "synopse":"Windows 8",
                  "tags":"windows 8",
                  "title":"windows_8",
                  "uid":"{...}",
                  "url":"http://fotos.sapo.pt/{...}/fotos/?uid={...}",
                  "user":{
                     "avatar":"http://imgs.sapo.pt/sapofotos/{...}/imgs/avatar.jpg",
                     "url":"http://fotos.sapo.pt/{...}/perfil",
                     "username":"{...}"
                  },
                  "views":{
                     "view":[
                        {
                           "requestHeight":"405",
                           "requestWidth":"540",
                           "size":"original",
                           "url":"http://{...}.jpeg"
                        },
                        (...)
                     ]
                  }
               },
               (...)
            ]
         },
         "result":{
            "ok":"true",
            "page":"1",
            "perPage":"50",
            "total":"41",
            "totalPages":"1"
         }
      }
   }
}
\end{lstlisting}


\subsection{ImageGetListBySearch}
\label{sec:chapter4:photos:http:imageGetListBySearch}

